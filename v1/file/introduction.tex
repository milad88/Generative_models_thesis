%%%%%%%%%%%%%%%%%%%%%%%%%%%%%%%%%%%%%%%%%%%%%%%%%%%%%%%%%%%
% Introduction
%\clearpage{\pagestyle{empty}}

\pagestyle{fancy}  
\chapter*{Introduction}
\addcontentsline{toc}{chapter}{Introduction}
\vspace{5 pt}
There is an immense effort in machine learning and statistics to develop accurate and scalable probabilistic models of data. Such models are called upon whenever we are faced with tasks requiring probabilistic reasoning, such as prediction, missing data imputation and uncertainty estimation; or in simulation-based analyses, common in many scientific fields such as genetics, robotics and control that require generating a large number of independent samples from the model. The benefit of employing a generative model lies in its wide applicability to the problem of inference. For example, many robotic tasks include the problem of evaluating the likelihood of an observation of the environment given some piece of relevant information, such as the location of the camera, a particular object model hypothesis, or even raw images. However, generative models in robotics are used to solve various issues faced during the implementation of applications in robotics field to design intelligent machines that can help and assist humans in their day-to-day lives. Such as learn from demonstrations or sometimes called imitation learning, which is a paradigm for enabling robots to autonomously perform new tasks, help reinforcement learning to learn from high-dimensional state space, lean wild set of tasks, and improve polices either with or without demonstrations. The lack of data as well is one of the main efficiency and stability is one of the main challenges in end-to-end methods like those of\textit{ \textacutedbl reinforcement learning\textgravedbl.} It is hard to talk about robotics without mention reinforcement learning, which as a field, has had the major success in past few years. RL offers to robotics a set of tools and algorithms that can assist to design complicated robotic behavior to achieve sophisticated tasks. More recently, another field in machine learning that had the lead in success well as is\textit{ \textacutedbl Generative Adversarial Networks\textgravedbl} this great idea of gaming networks, invented by Ian Goodfellow, has been exploit in RL to achieve more complicated frameworks that, as we will see later in this work.
\\During this work we will talk about the main techniques and algorithms used to perform generative models that have assisted the robotics applications to achieve the respective tasks, and go through some of the frameworks and papers that we deemed most interesting, and showed the importance role the generative modeling plays in the most recent robotics applications. Afterward we will discuss these frameworks explaining their functioning and how the generative models were employed to become useful in the various settings. Then we will show the results obtained from this employment, and argue the effects and the limits on these applications. At the end we will talk about the conclusions extracted during the extension of this survey, and offer some thoughts about how might the generative models in robotics extend and where would be the direction of the future works.


\clearpage{\pagestyle{empty}\cleardoublepage}
