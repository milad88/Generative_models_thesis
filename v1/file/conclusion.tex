%%%%%%%%%%%%%%%%%%%%%%%%%%%%%%%%%%%%%%%%%%%%%%%%%%%%%%%%%%%
% Conclusion

\pagestyle{fancy} 
\chapter*{Conclusion}
\addcontentsline{toc}{chapter}{Conclusion}
\vspace{-15pt}

The goal of this survey is to bring together some of the researches oriented for robots that make use of the knowledge about deep and probabilistic generative models to develop a future cognitive architecture. It also aims at examining the challenges and opportunities emerging from the interdisciplinary research field covering machine learning, and robotics. During this survey it was clear that the most commonly used and efficient approaches to achieve a generative models are Variational Autoencoders (VAE) we have seen in sec.~\ref{cha:VAE} and Generative Adversarial Networks (GAN) in sec.~\ref{cha:GANs}. Taking advantage of the ability of VAEs in either finding the latent space when dealing with high-dimensional input data, or to benefit of the its structure. VAE aims at maximizing the lower bound of the data log-likelihood. However, the skills that VAEs enjoy does not deny the efficiency of Gaussian Mixture Models (GMM) sec.~\ref{cha:GMM} in finding the probability distribution of the dataset from which data instances are drawn. GMMs were employed to encode and retrieval trajectories, and imitate demonstrated policies.\\ In a field like robotics the lack of data is one of the most common problems the researchers meet as I noticed in the different papers I have read, these researchers had to make the robot learn some task or imitate human behavior given a limited amount of demonstration, GMM has shown there ability in finding complex probability distribution. While the powerful of GANs is that can create new content based on guidance of the dataset. There are some issues usually are faced in training GANs is that they need large amount of data, the training process itself, and they struggle to find data distribution, even though they are powerful to generate visual data instances that are conform with the original dataset. GAN approach is to achieve an equilibrium between Generator and Discriminator, which make the training process confusing, where sometimes it hard to till if the model has reached what we called earlier the saddle point between the generator and discriminator.\\
The generative modeling in this survey has been done in different spaces as sensory space, more efficiently in latent representations of row observations, or to encode spatial features. The overall utility we found out is that these generative models helped to achieve more efficient frameworks in robotics field, this help was essentially obtained by generating more data for training and testing, or generate behaviors that are similar to an expert demonstration, giving chance to the agent to imitate the expert. This integration of the generative models in robotics applications, has almost opened a new field in machine learning, and gave birth to additional improvements to the various existing algorithms used in robotics, fastening their training process by reducing the number of epochs to train the model, when there are some demonstrations to learn from. technically speaking this integration is done by modifying the loss function both actor and the critic by adding the adversarial components. In addition, taking advantage of GANs has help to avoid to face the problem of formulate the reward function in RL. Finally, to conclude this work, I can tell that the generative model has been employed to either assist the existing approaches applied on robotics, or to introduce new ones that by taking advantage of the structures of the different instruments of generative modeling. However, this introduction has yielded a notable improvement to the robotics applications mainly by decreasing the training process where there is no more need to carry out the whole learning on only real world data, instead it is done using generated data instances. Moreover, the generative models made it easier for robot agents to clone a human behavior, by generating policies that are similar to the demonstrations made by humans, that consequently reduces the number of epochs the learning process should effectuate. Nevertheless, the generative model has of course helped to achieved better results for robotics applications, but still it is not enough though, the improvement margins are still wild. One of the main objectives of robotics researchers is to achieve robots able to perform wild variety of tasks, the generative modeling can help, though the path toward robotic manipulators that can execute a wide variety of tasks lies in multi-task learning as we have seen in~\cite{rahmatizadeh2018vision}. We can imagine that, in the future, a large number of end-users could specify their own tasks to their own robots, all what the users have to do is to show their robots some demonstrations of how to behave, with the help of the generative models, the robots could autonomously learn how to perform efficiently the tasks. Moreover, in future work, generative modeling might introduce opportunities for learning to observe not only from similar agents or humans, but also from other agents like robots with different embodiments whose actions are unknown or do not have a known correspondence, nevertheless, contribution would be to learn to transfer across environments.
\clearpage{\pagestyle{empty}\cleardoublepage}