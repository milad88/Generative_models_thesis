%%%%%%%%%%%%%%%%%%%%%%%%%%%%%%%%%%%%%%%%%%%%%%%%%%%%%%%%%%%
% Chapter3


\pagestyle{fancy} 
\chapter{Discussion}
\label{cha:3}
\vspace{1cm}

The goal of this survey is to bring together some of the researches oriented for robots that make use of the knowledge about deep and probabilistic generative models to develop a future cognitive architecture. It also aims at examining the challenges and opportunities emerging from the interdisciplinary research field covering machine learning, and robotics. During this survey it was clear that the most commonly used and efficient approaches to achieve a generative models are Variational Autoencoders (VAE) we have seen in sec.~\ref{cha:VAE} and Generative Adversarial Networks (GAN) in sec.~\ref{cha:GANs}. Taking advantage of the ability of VAEs in either finding the latent space when dealing with high-dimensional input data, or to benefit of the its structure. VAE aims at maximizing the lower bound of the data log-likelihood. However, the skills that VAEs enjoy does not deny the efficiency of Gaussian Mixture Models (GMM) sec.~\ref{cha:GMM} in finding the probability distribution of the dataset from which data instances are drawn. GMMs were employed to encode and retrieval trajectories, and imitate demonstrated policies. In a field like robotics the lack of data is one of the most common problems the researchers meet as well as I noticed in the different papers I have read, these researchers had to make the robot learn some task given a limited amount of demonstration or when it has to imitate human behavior. While the powerful of GANs is that can create new content based on guidance of the dataset. There are some issues usually are faced in training GANs is that they need large amount of data,  the training process itself, and they struggle to find data distribution, even though they are powerful to generate data instances that are conform with the original dataset. GAN approach is to achieve an equilibrium between Generator and Discriminator.  
The generative modeling in this survey has been done in different spaces as sensory space like in~\cite{nair2018visual}, more efficiently in latent representations of row observations~\cite{fujimoto2018addressing} , or to encode spatial features~\cite{finn2016deep}. However we will see more detailed examples in the next section. The overall utility we found out is that these generative models helped to achieve more efficient frameworks in robotics field, this help was essentially obtained by generating more data to for training and testing, or generate behaviors that are similar to an expert demonstration, giving chance to the agent to imitate the expert. This integration of the generative models in robotics applications, has almost opened a new field in machine learning, and gave birth to additional improvements to the various existing algorithms used in robotics, fastening their training process by reducing the number of data to train on when there are some demonstrations to learn from. Generative Adversarial Imitation Learning (GAIL)~\cite{DBLP:journals/corr/HoE16} is the best example to make that describe this integration. GAIL transforms the actor-critic approach of reinforcement learning to generative adversarial network, or vice versa, where the generator is acting as it is an actor the discriminator criticizes be distinguishing between authentic and generated data.

\clearpage{\pagestyle{empty}\cleardoublepage}