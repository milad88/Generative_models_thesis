%%%%%%%%%%%%%%%%%%%%%%%%%%%%%%%%%%%%%%%%%%%%%%%%%%%%%%%%%%%
% Chapter3


\pagestyle{fancy} 
\chapter{Reinforcement Learning (RL)}
\label{cha:3}
\vspace{1cm}

\section{What is RL}
This field of machine learning deals with how an agent ought to behave in an environment in order to maximize the reward. It differs from supervised learning in not needing of labeled input/output pairs and from unsupervised learning in getting guidance from the environment by performing actions and learning from the errors or rewards. Typically the environment take the form of a Markov Decision Process (MDP) is a mathematical system used for modeling decision making. We use a tuple (S, A, P, R, $\gamma$) to define a MDP. Where S denotes the state space, a finite set of states. A denotes a set of actions the actor can take at each time step t. P denotes the probability that taking action a at time step t in state st will result in state $s_{t+1}$. $R_a(s,\acute{s})$ is the expected reward from taking action a and transitioning to $\acute{s}$. $\gamma \in [0, 1]$ is a discount factor, to discount the future reward.

\vspace{0.3cm}
There are tow notions about the environment where the algorithm that implement RL that should be mentioned which are:
\begin{itemize}
\item “model-based” algorithms: who are employed when the environment is a priori known, in other words, when we know the transition probability matrix P between states, so the agent can make predictions about the next state and reward before it takes each action.
\end{itemize}
\begin{itemize}
\item “model-free” algorithms: for which there is no assumption about the world.
\end{itemize}
While about the techniques the algorithm uses to lean the policy are divided as follow:
\begin{itemize}
\item	“Off-policy”: is that it updates its Q-values using the Q-value of the next state s′ and the greedy action a′. In other words, it estimates the return (total discounted future reward) for state-action pairs assuming a greedy policy were followed despite the fact that it's not following a greedy policy.
\end{itemize}
\begin{itemize}
\item	“On policy”: is that it updates its Q-values using the Q-value of the next state s′ and the current policy's action a′′. It estimates the return for state-action pairs assuming the current policy continues to be followed.
\end{itemize}

In this work, all the algorithms referred to are model-free since in robotics applications usually the software agent can’t make any prediction about the environment, and no assumption is made whether it is on-policy or off-policy.

\vspace{0.3cm}
Going through the various algorithms of RL you can realize that in most cases there is not best algorithm, it all depends on task, environment, discrete or continuous spaces, and the data itself and its size. During my studies I have implemented different algorithms in RL which are Deep Q Learning (DQN), Deep Deterministic Policy Gradient (DDPG) and Trust Region Policy Optimization (TRPO). Basing on my modest experience I realized is that as long as we have simple and well-defined environment, and picking the algorithm whose more fit to the task taking into account the domain spaces of actions and states, you eventually will get good result, the agent will learn a close-to-optimal policy to behave in the environment. But when the task (policy) to be learned is more complicated in respect of the lack of resources and data and its quality, then it is more than convenient making some process on the input data to make the learning policy process more efficient computationally and of course in terms of results which are our aim first of all. That what I found out while doing my survey about generative models in robotics, where RL is strongly present regardless on which algorithm has been employed, actually most of time the algorithm used was not mentioned. 

\section{Exploitation of VAE and GMM in RL}
As mentioned in the previous section, most of the time applying RL algorithm directly on high-dimensional data does not lead to a good performance, so in the kind of situation it is advantageous to make use of the techniques that permit to reduce the dimensionality by representing the data in more suitable way.\\

One of the frameworks I went through has exploit both VAE and GMM to neatly which makes it feasible to fit the dynamics
even when the number of samples is much lower than the
dimensionality of the system. this what~\cite{finn2016deep} does, where initially RL algorithm run on robot with initial random policy to collect N (5 for that experiment) samples, then use them to fit GMM to learn the environment dynamics  or the policy controller without vision but using only the robot’s configuration as the state. In a second phase a VAE is trained to encode image dataset with unsupervised learning to produce a low-dimensional bottleneck vector that is a natural choice for learned feature representation or feature points for each image that concisely describes the configuration of objects in the scene, the interesting part of this VAE is that it is forced to encode spatial features rather than values. This is obtained  basically by applying spatial soft-max activation function that consists of tow operations on the last convolutional layer of the encoder as follow:
\begin{equation}
s_{cij} = \frac {e^{\frac{a_{cij}}{\alpha}}}
{\sum_{\acute{i}\acute{j}} e^{ \frac {a_{c \acute{i} \acute{j}}}{\alpha}}}
\end{equation}
where the temperature $\alpha$ is a learned parameter. Then, the expected
2D position of each softmax probability distribution $s_c$ is
computed according to:
\begin{equation}
f_c = (\sum_i i ∗ s_{cij} , \sum_j j ∗ s_{cij} )
\end{equation}
which forms the autoencoder's bottleneck and essentially it is the learned spatial feature point representation, that will therefore be capable of directly localizing objects in the image. The third and final phase of this framework is same as the first one, but the difference here is that the controller is trained on the feature points of the encoder using same trajectory-centric reinforcement
learning algorithm.\\
The experiments of this method showed that it could be used to learn a wild range of manipulation skills that require close coordination between perception and
action, and uses a spatial feature representation of the environment, which is learned as a bottleneck layer in
an autoencoder. This allows us to learn a compact state from high-dimensional real-world images. Furthermore, since this
representation corresponds to image-space coordinates of objects in the scene, it is particularly well suited for continuous control. The trajectory-centric RL algorithm we employ can learn a variety of manipulation skills with these spatial
representations using only tens of trials on the real robot.\\

In another work~\cite{nair2018visual} a RL framework was designed to jointly learns representations and policies from raw sensor inputs that achieve arbitrary goals under this representation by practicing to reach self-specified random goals during training. Here shows up the problem of choosing a suitable goal representations, to deal with this, a goal space \c{G} as to be same as the state space \c{S}. As well the problem of high-dimensional observations such as images arises, to handle it the authors once again the rely on VAE to learn a latent embedding for both \c{G} and \c{S}, by executing a random policy to collect state observation and optimize Eq.~\ref{eq:VAE_loss}, an additional online training has been introduced where the VAE is fine-tuned during the policy training each 3000 environment steps on all of the images observed by the policy, because as the policy improve it might visit new state observations where the VAE is not trained on, this additional training helped the performance of the overall algorithm. The final step is to run the RL algorithm which is a value-based one in this work twin delayed deep deterministic
policy gradients (TD3)~\cite{fujimoto2018addressing} is used, the thing that should be pointed out here that the negative Mahalanobis distance in the latent space were used as a reward function, but it turned out that minimizing this squared distance was equivalent to maximize the probability of the latent goal. This framework for learning general-purpose goal-conditioned policies that can achieve goals specified with target observations.\\




\clearpage{\pagestyle{empty}\cleardoublepage}