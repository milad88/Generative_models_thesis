%%%%%%%%%%%%%%%%%%%%%%%%%%%%%%%%%%%%%%%%%%%%%%%%%%%%%%%%%%%
% Introduction
%\clearpage{\pagestyle{empty}}

\pagestyle{fancy}  
\chapter*{Introduction}
\addcontentsline{toc}{chapter}{Introduction}

\vspace{5 pt}
intro
\section{Variational Auto-Encoder (VAE)}
\label{sec:}
One of the reasons for which the generative models have been employed in different type of applications is the powerful and utility of VAE, where they are used to either solve issues in AI like image reconstruction and generation, achieve better results, reduce the computational complexity due to the high dimensionality of the data, find latent space, reduce dimensionality, extract and represent features or learn density distribution of the dataset. In this session it is given an overview on how a VAE network is structured and what are the main techniques applied to make it useful to each of the issues just mentioned above.
\vspace{0.3cm}
\section{Gaussian mixture models (GMMs)}

GMM is a probabilistic model for representing normally distributed subpopulations within an overall population. Mixture models in general don't require knowing which subpopulation a data point belongs to, allowing the model to learn the subpopulations automatically. Since subpopulation assignment is not known, this constitutes a form of unsupervised learning. GMMs have been used for feature extraction from speech data, and have also been used extensively in object tracking of multiple objects, where the number of mixture components and their means predict object locations at each frame in a video sequence.



\clearpage{\pagestyle{empty}\cleardoublepage}
