%%%%%%%%%%%%%%%%%%%%%%%%%%%%%%%%%%%%%%%%%%%%%%%%%%%%%%%%%%%
% Chapter3


\pagestyle{fancy} 
\chapter{Generative Adversarial Networks (GANs)}
\label{cha:4}
\vspace{1cm}
\section{Whats a GAN}

Generative adversarial networks (GAN) is algorithmic architecture that uses two neural networks, pitting one against the other (thus the “adversarial”) in order to generate new, synthetic instances of data that can pass for real data. They are used widely in image generation, video generation and voice generation. it was introduced firstly by~\cite{goodfellow2014generative} to create a new framework for estimating generative models via an adversarial process that corresponds to tow-player game,
the tow networks could have arbitrary architecture and they are trained simultaneously, one neural network, called the discriminator, is designed as classifier network to evaluate the authenticity  distinguishing between fake and real data instances, while the other one, called generator, is trained to generate data as close to the authentic ones. Meanwhile, the generator is creating new, synthetic instances that it passes to the discriminator. It does so in the hopes that they, too, will be deemed authentic, even though they are fake. The goal of the generator is to generate passable instances to lie without being caught. The goal of the discriminator is to identify those coming from the generator as fake.

\section{Generative vs. Discriminative Algorithms}
